\documentclass[uplatex]{jsarticle}
\makeatletter
  \usepackage{lmodern}%(Latin Modern font)
  \usepackage[T1]{fontenc}
  \usepackage{amsmath}
  \usepackage{amssymb}
  \usepackage{bm}
  \usepackage[dvipdfmx]{xcolor}
  \usepackage[dvipdfmx]{graphicx}
  \usepackage{apmayfes2015}
  \usepackage{apmayfes2015deco}
% ---- 必要なパッケージは適宜追加してください ----
  \usepackage{ascmac}
  \usepackage{url}
  \usepackage{tikz}
  \usepackage{listings}
  \usepackage{subcaption}	% subfigure emvironment(horizontal assignment of figure)
  \usepackage[english]{babel}
% ---- ここに独自のマクロの定義を書きましょう ----
  \def\catC{\mathbf{C}}
  \def\functIdemC{\mathrm{Id}_{\catC}}
  \newcommand{\If}{ \ \mathrm{if} \ }
  \newcommand{\Where}{ \ \mathrm{where} \ }
  \newcommand{\tr}{\mathop{\mathrm{tr}}}
  \newcommand{\mathhyphen}{\mathchar`-}
  % color define section
\definecolor{mygreen}{rgb}{0,0.6,0}
\definecolor{mygray}{rgb}{0.5,0.5,0.5}
\definecolor{mymauve}{rgb}{0.58,0,0.82}
%
%listings set section
\lstset{ %
  backgroundcolor=\color{white},   % choose the background color; you must add \usepackage{color} or \usepackage{xcolor}
  basicstyle=\footnotesize\ttfamily,        % the size of the fonts that are used for the code
  breakatwhitespace=false,         % sets if automatic breaks should only happen at whitespace
  breaklines=true,                 % sets automatic line breaking
  captionpos=b,                    % sets the caption-position to bottom
  commentstyle=\color{blue},    % comment style
  deletekeywords={...},            % if you want to delete keywords from the given language
  escapeinside={\%*}{*)},          % if you want to add LaTeX within your code
  extendedchars=true,              % lets you use non-ASCII characters; for 8-bits encodings only, does not work with UTF-8
  frame=single,                    % adds a frame around the code
  keepspaces=true,                 % keeps spaces in text, useful for keeping indentation of code (possibly needs columns=flexible)
  keywordstyle=\color{mygreen},       % keyword style
  language=C,                 % the language of the code
  morekeywords={*,...},            % if you want to add more keywords to the set
  numbers=left,                    % where to put the line-numbers; possible values are (none, left, right)
  numbersep=5pt,                   % how far the line-numbers are from the code
  numberstyle=\tiny\color{mymauve}, % the style that is used for the line-numbers
  rulecolor=\color{black},         % if not set, the frame-color may be changed on line-breaks within not-black text (e.g. comments (green here))
  showspaces=false,                % show spaces everywhere adding particular underscores; it overrides 'showstringspaces'
  showstringspaces=false,          % underline spaces within strings only
  showtabs=false,                  % show tabs within strings adding particular underscores
  stepnumber=2,                    % the step between two line-numbers. If it's 1, each line will be numbered
  stringstyle=\color{mymauve},     % string literal style
  tabsize=2,                       % sets default tabsize to 2 spaces
  title=\lstname                   % show the filename of files included with \lstinputlisting; also try caption instead of title
}
%
\graphicspath{{data03/fig/},{data04/fig/}}	% path of includegraphics
% ---- ----
%
  \title{Problem 03}
  \author{140588 Takashi Kurokawa}
\makeatother
\begin{document}
  \maketitle
%
  \section{Abstruct}
	\indent
	  Estimated parameters of a mixture gaussian distribution data come from, using EM algrithm. I implemented program with C++, processed data with Python, analyzed data with R. 
	\par
%
  \section{EM algorithm}
    \subsection{parameters update}
      \indent
	See lecture handout.
      \par
%
  \section{Numerical experiment 1}
    \indent
      I estimated parameters of the true distribution by EM algorithm with: \textbf{Fixed true distribution, fixed tutor data, various initial parameters.} I set dimension $2$, number of mixed gaussian distributions $4$.
    \par
    \subsection{true parameters}
      \indent
	I set true parameters below:
	\begin{description}
	  \item[mixture rate] $\pi_1 = 0.1 : \pi_2 =  0.2 : \pi_3 = 0.3 : \pi_4 = 0.4$
	  \item[mean vectors] $\mu_1 = [-2,-2], \mu_2 = [-2,2], \mu_3 = [2,-2], \mu_4 = [2,2]$
	  \item[covariance matrices]
	    $
	      \Sigma_1 = \Sigma_2 = \Sigma_3 = \Sigma_4 = I_2 =
	      \begin{pmatrix}
		1 & 0
		\\
		0 & 1
	      \end{pmatrix}
	    $
	\end{description}
	See Figure \ref{fig:true_contour} and Figure \ref{fig:true_persp} also.
	\begin{figure}[h]
	\centering
	  \begin{subfigure}{0.48\columnwidth}
	  \centering
	  \includegraphics[width=\columnwidth]{contour.pdf}
	  \caption{contour}
	  \label{fig:true_contour}
	  \end{subfigure}
	  \begin{subfigure}{0.48\columnwidth}
	  \centering
	  \includegraphics[width=\columnwidth]{persp.pdf}
	  \caption{perspective}
	  \label{fig:true_persp}
	  \end{subfigure}
	\caption{true distribution}
	\label{fig:true}
	\end{figure}
      \par
    \subsection{Tutor data}
      \indent
	I generated 1000 tutor data with R. See Figure \ref{fig:square1000}.
	\begin{figure}[h]
	\centering
	\includegraphics[width=0.96\columnwidth]{square1000.pdf}
	\caption{1000 tutor data}
	\label{fig:square1000}
	\end{figure}
      \par
    \subsection{Initial parameters}
      \indent
	I generated 6000 initial parameters as below:
	\begin{description}
	  \item[mixture rate] First I generated 4 real numbers uniform distribution in $[0,1]$. Second I normalize them so that their sum is equal to 1 and set them as mixture rate. Repeated it 6000 times.
	  \item[mean vectors] I generated 8 components of 4 vectors from uniform distribution in $[-10,10]$ and set them as mean vectors. Repeated it 6000 times.
	  \item[covariance matrices] First I generated 16 components of 4 matrices from 6 gaussian distributions - 1000 from $N_1(0,1)$, 1000 from $N_1(0,4)$, 1000 from $N_1(0,25)$, 1000 from $N_1(0,0.25)$, 1000 from $N_1(0,100)$, 1000 from $N_1(0,2500)$. If the generated matrix is not invertible, regenerated. Second I produce each generated matrix and its tranposition, got 4 positive-definite symmetrical matrices, and set them as convariance matrices.
	\end{description}
      \par
    \subsection{Stop condition}
      \indent
	I repeated EM algorithm until differences of all parameters and a logical likelifood through a step are greater than $\epsilon = 10^{-6}$, very small positive number. In addition, the program skip test with specific initial values in two cases: [1] In parameters update phase, some covariant matrix is not invertible(because of numerical error). [2] In logical likelifood calculate phase, probability distribution function with some tutor data is so small that $\log (\pi_i * \app{\mathrm{pdf}}{x_t})$ is evaluated $-\infty$.
      \par
    \subsection{Estimate}
      \indent
	I tried explaining convergence and its speed of EM algorithm with initial parameters. I constructed the following assumpotion:
      \par
      \indent
	\begin{shadebox}
	  \begin{itemize}
	    \item The logical likelifood with initial parameters reflects similarity between initial mixture gaussian ditribution and true distribution. Therefore that explain convergence and its speed.
	    \item Jensen-Shannon divergence among mixed initial gaussian distributions reflects flexibility initial gaussian mixture ditribution. Therefore that explain convergence and its speed.
	  \end{itemize}
	\end{shadebox}
      \par
      \indent
	Jensen-Shannon divergence(JS divergence from now on) is commutative Kullback-Leibler devergence(KL divergence from now on). KL divergence of multi\_variate normal distribution is below:
	\begin{align}
	  \app{KL}{X_1, X_2} = \frac{1}{2}\paren{\log\frac{\absprn{\Sigma_2}}{\absprn{\Sigma_1}} + \tr\paren{\Sigma_2^{-1}\Sigma_1} + \trsps{\paren{\mu_1 - \mu_2}}\Sigma_2^{-1}\paren{\mu_1 - \mu_2} - d}
	\end{align}
	JS divergence below:
	\begin{align}
	  \app{JS}{X_1, X_2} = \frac{1}{4}\paren{\tr\paren{\Sigma_2^{-1}\Sigma_1} + \tr\paren{\Sigma_1^{-1}\Sigma_2} + \trsps{\paren{\mu_1 - \mu_2}}\paren{\Sigma_1^{-1} + \Sigma_2^{-1}}\paren{\mu_1 - \mu_2} - 2d}
	\end{align}
      \par
    \subsection{Result and analysis}
      \indent
	First I removed records whose logical likelifood is $-\infty$ and devided records into two groups, Group 1 \textemdash param-converging and Group 2 \textemdash param-diverging. I made a two-sample Kolmogorov-Smirnov test for their mean of JS divergences between initial mixed distributions(JS mean), max of JS divergences between initial mixed distributions(JS max), initial logical likelifood(logL).
      \par
      \indent
	According to the tests, all three variables are two-sided between Group 1 and 2. See Figure \ref{fig:condi} also.
	\begin{figure}[h]
	\centering
	  \begin{subfigure}{0.32\columnwidth}
	  \centering
	  \includegraphics[width=\columnwidth]{loglogLcondi.pdf}
	  \caption{log(logL)}
	  \label{fig:loglogLcondi}
	  \end{subfigure}
	  \begin{subfigure}{0.32\columnwidth}
	  \centering
	  \includegraphics[width=\columnwidth]{logKLmeancondi.pdf}
	  \caption{log(JSmean)}
	  \label{fig:logJSmeancondi}
	  \end{subfigure}
	  \begin{subfigure}{0.32\columnwidth}
	  \centering
	  \includegraphics[width=\columnwidth]{logKLmaxcondi.pdf}
	  \caption{log(JSmax)}
	  \label{fig:logJSmaxcondi}
	  \end{subfigure}
	\caption{boxplot \textemdash converges or not}
	\label{fig:condi}
	\end{figure}
      \par
      \indent
	Second I removed records in Group 2 and devided records into two groups again, Group 1' \textemdash settled in local optimum and Group 2' \textemdash settled in global optimum. I made a two-sample Kolmogorov-Smirnov test for the same three variables.
      \par
      \indent
	All three variables are two-sided between Group 1' and 2' again. See Figure \ref{fig:localunlocal} also.
	\begin{figure}[h]
	\centering
	  \begin{subfigure}{0.32\columnwidth}
	  \centering
	  \includegraphics[width=\columnwidth]{loglogLlocalunlocal.pdf}
	  \caption{log(logL)}
	  \label{fig:loglogLlocalunlocal}
	  \end{subfigure}
	  \begin{subfigure}{0.32\columnwidth}
	  \centering
	  \includegraphics[width=\columnwidth]{logKLmeanlocalunlocal.pdf}
	  \caption{log(JSmean)}
	  \label{fig:logJSmeanlocalunlocal}
	  \end{subfigure}
	  \begin{subfigure}{0.32\columnwidth}
	  \centering
	  \includegraphics[width=\columnwidth]{logKLmaxlocalunlocal.pdf}
	  \caption{log(JSmax)}
	  \label{fig:logJSmaxlocalunlocal}
	  \end{subfigure}
	\caption{boxplot \textemdash locally settled or not}
	\label{fig:localunlocal}
	\end{figure}
      \par
      \indent
	Now we can conclude: JS mean, JS max, and logL, which are all charactoristics of initial parameters, are related whether EM converges or not, and if it does, whether EM settles in global optimum or not.
      \par
%
  \section{Numerical experiment 2}
    \indent
      I estimated parameters of the true distribution by EM algorithm with: \textbf{Fixed true distribution, various tutor data, fixed initial parameters.} I set dimension $2$, number of mixed gaussian distributions $4$.
    \par
    \subsection{true parameters}
      \indent
	I set true parameters same as experiment 1.
      \par
    \subsection{Tutor data}
      \indent
	I generated 100, 500, 1000 tutor data with R. See Figure \ref{fig:square1000}.
	\begin{figure}[h]
	\centering
	  \begin{subfigure}{0.32\columnwidth}
	  \centering
	  \includegraphics[width=\columnwidth]{square100.pdf}
	  \caption{size 100}
	  \label{fig:square100}
	  \end{subfigure}
	  \begin{subfigure}{0.32\columnwidth}
	  \centering
	  \includegraphics[width=\columnwidth]{square1000.pdf}
	  \caption{size 1000}
	  \label{fig:square1000}
	  \end{subfigure}
	  \begin{subfigure}{0.32\columnwidth}
	  \centering
	  \includegraphics[width=\columnwidth]{square5000.pdf}
	  \caption{size 5000}
	  \label{fig:square5000}
	  \end{subfigure}
	\caption{tutor data}
	\label{fig:condi}
	\end{figure}
      \par
    \subsection{Initial parameters}
      \indent
	I generated 1000 initial parameters as below:
	\begin{description}
	  \item[mixture rate] Exactly same as experiment 1.
	  \item[mean vectors] Exactly same as experiment 1.
	  \item[covariance matrices] First I generated 16 components of 4 matrices from a gaussian distribution $N_1(0,1)$ and get covariance matrices in the same way as experiment 1.
	\end{description}
      \par
    \subsection{Stop condition}
      \indent
	Exactly same as experiment 1.
      \par
    \subsection{Estimate}
      \indent
	I tried explaining convergence and its speed of EM algorithm with size of tutor data. I constructed the following assumpotion:
      \par
      \indent
	\begin{shadebox}
	  \begin{itemize}
	    \item The more tutor data you give, the more often EM algorithm converges, gets to the global optimum and the faster it does.
	  \end{itemize}
	\end{shadebox}
      \par
    \subsection{Result and analysis}
      \indent
	First I assumed: Sizes of tutor data influence convergence of EM algorithm(alternative hypothesis $H_1$). So I conducted independence test between size of tutor data and whether EM converges or not. According to the test, they are not independent(See Table \ref{tab:SandC} also).
	\begin{table}[h]
	  \centering
	  \caption{Chi-square test(size and convergence)}
	  \begin{tabular}{lrrr}
	    \hline
	    & 100 & 1000 & 5000
	    \\ \hline
	    converge & 106 & 252 & 320
	    \\
	    diverge & 894 & 748 & 680
	    \\ \hline
	    \rule{0pt}{1em}
	    \\ \hline \hline
	    \multicolumn{4}{c}{$p = 2.071e\mathhyphen 30$}
	    \\ \hline \hline
	  \end{tabular}
	  \label{tab:SandC}
	\end{table}
      \par
      \indent
	Second I assumed: The larger tutor data we use, the more rapidly EM algorithm converges(alternative hypothesis $H_1$). So I extracted converged cases and applied ANOVA, size of tutor data to the step EM converges(See Figure \ref{fig:anovaconverge}).
	\begin{figure}[h]
	\centering
	  \begin{subfigure}{0.48\columnwidth}
	  \centering
	  \includegraphics[width=\columnwidth]{sizeconvergeRF.pdf}
	  \caption{Residuals vs Fitted}
	  \label{fig:sizeconvergeRF}
	  \end{subfigure}
	  \begin{subfigure}{0.48\columnwidth}
	  \centering
	  \includegraphics[width=\columnwidth]{sizeconvergeQQ.pdf}
	  \caption{Normal Q-Q}
	  \label{fig:sizeconvergeQQ}
	  \end{subfigure}
	  \begin{subfigure}{0.48\columnwidth}
	  \centering
	  \includegraphics[width=\columnwidth]{sizeconvergeSL.pdf}
	  \caption{Scale-Location}
	  \label{fig:sizeconvergeSL}
	  \end{subfigure}
	  \begin{subfigure}{0.48\columnwidth}
	  \centering
	  \includegraphics[width=\columnwidth]{sizeconvergeCook.pdf}
	  \caption{Residuals vs Leverage}
	  \label{fig:sizeconvergeCook}
	  \end{subfigure}
	\caption{ANOVA(size and converging speed)}
	\label{fig:anovaconverge}
	\end{figure}
	At first look at figures, we realize that this model poorly explains data. Each group doesn't follow any gaussian distributions obviously, and therefore I test the same hypothesis with Kruskal-Wallis test(See Table \ref{tab:SandS})
	\begin{table}[h]
	  \centering
	  \caption{Kruskal-Wallis test(size and converging speed)}
	  \begin{tabular}{c}
	    \hline \hline
	    $p = 9.362e\mathhyphen 07$
	    \\ \hline \hline
	  \end{tabular}
	  \label{tab:SandS}
	\end{table}
	Now we know that large tutor data accelerate EM convergence.
      \par
      \indent
	Third I assumed: Sizes of tutor data influence settlement in the global optimum of EM algorithm(alternative hypothesis $H_1$). So I extracted converged cases and conducted independence test between size of tutor data and whether EM settles in the global optimum or not(just a local optimum). According to the test, they are not independent(See Table \ref{tab:SandG} also).
	\begin{table}[h]
	  \centering
	  \caption{Chi-square test(size and global settlement)}
	  \begin{tabular}{lrrr}
	    \hline
	    & 100 & 1000 & 5000
	    \\ \hline
	    global & 27 & 160 & 211
	    \\
	    local & 79 & 92 & 109
	    \\ \hline
	    \rule{0pt}{1em}
	    \\ \hline \hline
	    \multicolumn{4}{c}{$p = 3.141e\mathhyphen 13$}
	    \\ \hline \hline
	  \end{tabular}
	  \label{tab:SandG}
	\end{table}
      \par
      \indent
	Fourth I assumed: The larger tutor data we use, the more rapidly EM algorithm converges into the global optimum(alternative hypothesis $H_1$). So I extracted global-optimum converged cases and applied ANOVA, size of tutor data to the step EM converges(See Figure \ref{fig:anovaglobal}).
	\begin{figure}[h]
	\centering
	  \begin{subfigure}{0.48\columnwidth}
	  \centering
	  \includegraphics[width=\columnwidth]{sizeglobalRF.pdf}
	  \caption{Residuals vs Fitted}
	  \label{fig:sizeglobalRF}
	  \end{subfigure}
	  \begin{subfigure}{0.48\columnwidth}
	  \centering
	  \includegraphics[width=\columnwidth]{sizeglobalQQ.pdf}
	  \caption{Normal Q-Q}
	  \label{fig:sizeglobalQQ}
	  \end{subfigure}
	  \begin{subfigure}{0.48\columnwidth}
	  \centering
	  \includegraphics[width=\columnwidth]{sizeglobalSL.pdf}
	  \caption{Scale-Location}
	  \label{fig:sizeglobalSL}
	  \end{subfigure}
	  \begin{subfigure}{0.48\columnwidth}
	  \centering
	  \includegraphics[width=\columnwidth]{sizeglobalCook.pdf}
	  \caption{Residuals vs Leverage}
	  \label{fig:sizeglobalCook}
	  \end{subfigure}
	\caption{ANOVA(size and converging speed(global optimum))}
	\label{fig:anovaglobal}
	\end{figure}
	Observing figures, we realize that data includes some outliers. I remove records 144, 344, 365 as outliers and reapplied ANOVA(See Figure \ref{fig:anovaglobalmod}).
	\begin{figure}[h]
	\centering
	  \begin{subfigure}{0.48\columnwidth}
	    \centering
	    \includegraphics[width=\columnwidth]{sizeglobalRFmod.pdf}
	    \caption{Residuals vs Fitted}
	    \label{fig:sizeglobalRFmod}
	  \end{subfigure}
	  \begin{subfigure}{0.48\columnwidth}
	    \centering
	    \includegraphics[width=\columnwidth]{sizeglobalQQmod.pdf}
	    \caption{Normal Q-Q}
	    \label{fig:sizeglobalQQmod}
	  \end{subfigure}
	  \begin{subfigure}{0.48\columnwidth}
	    \centering
	    \includegraphics[width=\columnwidth]{sizeglobalSLmod.pdf}
	    \caption{Scale-Location}
	    \label{fig:sizeglobalSLmod}
	  \end{subfigure}
	  \begin{subfigure}{0.48\columnwidth}
	    \centering
	    \includegraphics[width=\columnwidth]{sizeglobalCookmod.pdf}
	    \caption{Residuals vs Leverage}
	    \label{fig:sizeglobalCookmod}
	  \end{subfigure}
	\caption{reANOVA(size and converging speed(global optimum))}
	\label{fig:anovaglobalmod}
	\end{figure}
	Q-Q plot implies that an error doesn't follow any gaussian distribution. I confirmed it with Shapiro-Wilk test. The result is below:
	\begin{table}[h]
	  \centering
	  \caption{Shapiro-Wilk test(normality of error)}
	  \begin{tabular}{c}
	    \hline \hline
	    $p = 2.2e\mathhyphen 16$
	    \\ \hline \hline
	  \end{tabular}
	  \label{tab:SandS}
	\end{table}
	It says that error has no normality, meaning that I can't apply ANOVA to this data. I took Kruskal-Wallis test I conducted with the second hypothesis.
	\begin{table}[h]
	  \centering
	  \caption{Kruskal-Wallis test(size and converging speed)}
	  \begin{tabular}{c}
	    \hline \hline
	    $p = 7.692e\mathhyphen 15$
	    \\ \hline \hline
	  \end{tabular}
	  \label{tab:SandS}
	\end{table}
	Now we know that large tutor data accelerate EM convergence into the global optimum.
      \par
      \indent
	Now I realize that my prediction about relation between the size of tutor data and EM performances(whether it converges or not, settles in global optimum or not, how rapidly it converges). Large data set will boost power of EM algorithm.
      \par
%
\end{document}
